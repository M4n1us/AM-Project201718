\documentclass[11pt,a4paper]{article}
\usepackage[utf8]{inputenc}
\usepackage{amsmath}
\usepackage{amsfonts}
\usepackage{amssymb}
\author{Barbara Wiedermann}
\newcommand\tab[1][1cm]{\hspace*{#1}}
\begin{document}
	\subsection{Bsp:}
	c) a\textsubscript{n} = (-1)\textsuperscript{n+1} * $\frac{3}{7n\textsuperscript{2} + 3}$\\
	 \tab $\epsilon$ = $\frac{1}{40}$\\\\
	
	\(\lim\limits_{n \to \infty}\ a\textsubscript{n}=?\)\\
	
	\(\lim\limits_{n \to \infty}\ (-1)\textsuperscript{n+1} * \frac{3}{7n\textsuperscript{2} + 3}= \lim\limits_{n \to \infty}\ (-1)\textsuperscript{n+1} * \)\(\lim\limits_{n \to \infty}\frac{3}{7n\textsuperscript{2} + 3}\) = 0\\\\\\
	$\Bigl| a\textsubscript{n} - a\Bigl| $ $<$ $\epsilon$\\\\\\
	$\Bigl| (-1)\textsuperscript{n+1} *$ $\frac{3}{7n\textsuperscript{2} + 3}$ $\Bigl| $ \\\\\\
	Fall Unterscheidung:\\\\
	1.Fall: n ... grade\\\\
	$\Bigl|$ - $\frac{3}{7n\textsuperscript{2} + 3}$ $\Bigl| $ $<$ $\frac{1}{40}$\\\\\\
	120 $<$ 7n\textsuperscript{2} + 3\\
	117 $<$ 7n\textsuperscript{2}\\
	$\frac{117}{7}$ $<$ n\textsuperscript{2}\\
	n $>$ $\sqrt{\frac{117}{7}}$\\\\
	
	Antwort: Für n gerade sind bis zum 7ten Glied alle Folfeflierder ausßerhalb der $\epsilon$ - Umgebung.\\\\\\
	
	2.Fall: n ... ungerade\\\\
	$\Bigl|$ $\frac{3}{7n\textsuperscript{2} + 3}$ $\Bigl|$ $<$ $\frac{1}{40}$ wie oben
	
	
	
	
	
	
\end{document}