\documentclass[../mainfile.tex]{subfile}
\author{Barbara Wiedermann}
\newcommand\tab[1][1cm]{\hspace*{#1}}
\begin{document}
\subsection*{Bsp:}
$a_n = (-1)^{n+1} * \frac{3}{7n^2 + 3}$\\
$\epsilon$ = $\frac{1}{40}$\\\\
$\lim\limits_{n \to \infty} a_n = ?$\\\\
$\lim\limits_{n \to \infty} (-1)^n+1 * \frac{3}{7n^2 + 3} = \lim\limits_{n \to \infty} (-1)^n+1 * \lim\limits_{n \to \infty}\frac{3}{7n^2 + 3} = 0$\\\\
$\abs*{a_n - a} < \epsilon$\\\\
$\abs*{ (-1)^{n+1} * \frac{3}{7n^2 + 3}} $ \\\\\\
Fall Unterscheidung:\\
1.Fall: n ... grade\\
$\abs*{ - \frac{3}{7n^2 + 3}} < \frac{1}{40}$\\\\\\
$120 < 7n^2 + 3$\\
$117 < 7n^2$\\
$\frac{117}{7} < n^2$\\
$n > \sqrt{\frac{117}{7}}$\\\\
Antwort: Für n gerade sind bis zum 7ten Glied alle Folfeflierder ausßerhalb der $\epsilon$-Umgebung.\\
2.Fall: n ... ungerade\\\\
$\abs*{\frac{3}{7n^2 + 3}} < \frac{1}{40}$ wie oben

\end{document}