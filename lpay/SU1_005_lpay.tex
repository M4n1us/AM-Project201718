\documentclass[../mainfile.tex]{subfiles}

\begin{document}
	
	\section{Grenzwertsätze}
	\subsection{Definition:}
		Seien $a_n$ und $b_n$ Folgen, sowie $\lambda \in \mathbb{R}$
		
		\begin{enumerate}
			\item[i)] Eine Folge besitzt höchstens einen Grenzwert. \\
			\item[ii)] Jede Folge, die konvergiert, ist notwendigerweise beschränkt.\\
			\item[iii)] Sei $\displaystyle \lim_{n\to\infty} a_n = a$ und $\displaystyle\lim_{n\to\infty} b_n = b$
			\begin{enumerate}[label=(\alph*)]
				\item $\displaystyle\lim_{n\to\infty} (a_n+b_n) = a+b$
				\item $\displaystyle\lim_{n\to\infty} (\lambda a_n) = \lambda a$
				\item $\displaystyle\lim_{n\to\infty} (a_n \cdot b_n) = a \cdot b$
				\item Falls $b\neq 0$ \\
				$\displaystyle\lim_{n\to\infty} (\frac{a_n}{b_n}) = \frac{a}{b}$
			\end{enumerate}
			\item Ist $a_n$ konvergent gegen a und $a_n \geqslant c \forall n \in \mathbb{N}$, dann ist auch $a \geqslant c$. \\
			Analog für $a_n \leqslant c$  
		\end{enumerate}
	
	\section{Sandwich-Lemma:} 
		Sein $a_n$ und $b_n$ zwei reelle konvergente Folgen mit dem selben Grenzwert a \\(also $\displaystyle \lim_{n\to\infty} a_n = a$ und $\displaystyle\lim_{n\to\infty} b_n = a$) \\
		so gilt: $a_n \leqslant c_n \leqslant b_n$, dass $\displaystyle\lim_{n\to\infty} c_n = a$ 
	\subsection*{Bsp:}
		$a_n = \sqrt[n]{4^n+7^n}$ \\  
		sicher kleiner: $\sqrt[n]{7^n}$ \\
		sicher größer: $\sqrt[n]{7^n+7^n}$ \\\\
		$\underbrace{\sqrt[n]{7^n}}_{\text{7}} \geqslant \sqrt[n]{4^n+7^n} \geqslant \underbrace{\sqrt[n]{7^n+7^n}}_{7 \cdot \sqrt[n]{2} = 7}$ \\
		$\Rightarrow \displaystyle\lim_{n\to\infty} \sqrt[n]{4^n+7^n} = 7$
		
\end{document}